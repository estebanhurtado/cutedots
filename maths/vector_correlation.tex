\documentclass[12pt]{article}
\begin{document}

\title{}
\author{}
\date{}
\maketitle

\section{Motion vectors correlation}

Here we show how to compute a correlation $r$ for two sequences of vectors, $\vec{a}$ and $\vec{b}$, so that the result is a scalar between $-1$ and $1$. It should tend towards $r=1$ when corresponding vectors point to a similar direction, and towards $r=-1$ when pointing to opposite directions instead. It should be close to $r=0$ when no such relationship exists between the two sequences.

Let $\vec{a}_i = (a_{i,1}, a_{i,2}, a_{i,3})$ for $i=1..n$, and $\vec{b}_i = (b_{i,1}, b_{i,2}, b_{i,3})$ for $i=1..n$. For implicity, we asume that
	\[ \vec{\mu}_a = \sum_{i=1}^n \vec{a}_i = \vec{0} \]
	\[ \vec{\mu}_b = \sum_{i=1}^n \vec{b}_i = \vec{0} \]

We define $r_{ab}$, the correlation between $\vec{a}$ and $\vec{b}$, as follows.

\[ r_{ab} = \frac
	{ \sum_{i=1}^n (\vec{a}_i - \vec{\mu}_b) \cdot (\vec{b}_i - \vec{\mu}_b) }
	{ 	\sqrt{\sum_{i=1}^n |\vec{a}_i -  \vec{\mu}_a|^2}
		\sqrt{\sum_{i=1}^n |\vec{b}_i -  \vec{\mu}_b|^2}} \]
		
\[ = \frac
	{ \sum_{i=1}^n \vec{a}_i \cdot \vec{b}_i }
	{ 	\sqrt{\sum_{i=1}^n |\vec{a}_i|^2}
		\sqrt{\sum_{i=1}^n |\vec{b}_i|^2}} \]
		
Note that
\[ r_{ab} = \left\{
	\begin{array}{ll}
		r_{ab} = 1 & \mbox{if } \vec{a} = \vec{b} \\
		r_{ab} = -1 & \mbox{if } \vec{a} = -\vec{b} \\
		-1 < r_{ab} < 1 & \mbox{otherwise}  
	\end{array}
\right. \]
	
Also,
\[ r_{ab} = \frac
	{ \sum_{i=1}^n (a_{i,1} b_{i,1} + a_{i,2} b_{i,2} + a_{i,3} b_{i,3}) }
	{ 	\sqrt{\sum_{i=1}^n (a_{i,1}^2 + a_{i,2}^2 + a_{i,3}^2) }
		\sqrt{\sum_{i=1}^n (b_{i,1}^2 + b_{i,2}^2 + b_{i,3}^2)} } \]
		

Therefore if components of $\vec{a}$ and $\vec{b}$ are uncorrelated we have
\[ \mathbf{E}\bigg(\sum_{i=1}^n  a_{i,j}b_{i,j} \bigg) = 0 \mbox{ for } j=1..3 \]

\[ \mathbf{E}(r_{ab}) = 0 \]

Finally, in order to compute a correlation we have
\[ r_{ab} = \frac
	{ \sum_{i=1}^n \sum_{j=1}^3 a_{i,j} b_{i,j} }
	{ 	\sqrt{\sum_{i=1}^n \sum_{j=1}^3 a_{i,j}^2 }
		\sqrt{\sum_{i=1}^n \sum_{j=1}^3 b_{i,j}^2 } } \]

\section{Combining correlations from several sequences}

The expression for $r_{ab}$ in the previous section can be written like this:

\[ r_{ab} = \frac {s_{ab}} {\sqrt{ss_a  ss_b}} \]

Where
\[ s_{ab} = \sum_{i=1}^n \vec{a}_i \cdot \vec{b}_i \]
\[ ss_a = \sum_{i=1}^n |\vec{a}_i|^2 \]
\[ ss_b = \sum_{i=1}^n |\vec{b}_i|^2 \]

In short $ss_a$ is the sum of squares of $\vec{a}$, $ss_b$ is the same for $\vec{b}$, and $s_{ab}$ is the sum of products. We could append more $ab$ data pairs from additional sequences this way

\[ r_{ab} = \frac {s_{ab,1} + s_{ab,2} + ... + s_{ab,n}}
	{\sqrt{(ss_{a,1} + ss_{a,2} + ... + ss_{a,n})(ss_{b,1} + ss_{b,2} + ... + ss_{b,n})}} \]

only keeping in mind that residuals in each $(s_{ab}, ss_a, ss_b)$ trio are subject to the restriction of having a zero sum, which means there are $n-1$ degrees of freedom times $3$ vector components. The total degrees of freedom for a combined $r_{ab}$ of $m$ sequences would be

\[ df = \sum_{j=1}^m (3*n_j - 3) \]

\end{document}
